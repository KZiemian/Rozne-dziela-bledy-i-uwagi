% Autor: Kamil Ziemian

% ---------------------------------------------------------------------
% Podstawowe ustawienia i pakiety
% ---------------------------------------------------------------------
\RequirePackage[l2tabu, orthodox]{nag}  % Wykrywa przestarzałe i niewłaściwe
% sposoby używania LaTeXa. Więcej jest w l2tabu English version.
\documentclass[a4paper,11pt]{article}
% {rozmiar papieru, rozmiar fontu}[klasa dokumentu]
\usepackage[MeX]{polski}  % Polonizacja LaTeXa, bez niej będzie pracował
% w języku angielskim.
\usepackage[utf8]{inputenc}  % Włączenie kodowania UTF-8, co daje dostęp
% do polskich znaków.
\usepackage{lmodern}  % Wprowadza fonty Latin Modern.
\usepackage[T1]{fontenc}  % Potrzebne do używania fontów Latin Modern.



% ---------------------------------------
% Podstawowe pakiety (niezwiązane z ustawieniami języka)
% ---------------------------------------
\usepackage{microtype}  % Twierdzi, że poprawi rozmiar odstępów w tekście.
% \usepackage{graphicx}  % Wprowadza bardzo potrzebne komendy do wstawiania
% % grafiki.
% \usepackage{verbatim}  % Poprawia otoczenie VERBATIME.
% \usepackage{textcomp}  % Dodaje takie symbole jak stopnie Celsiusa,
% % wprowadzane bezpośrednio w tekście.
\usepackage{vmargin}  % Pozwala na prostą kontrolę rozmiaru marginesów,
% za pomocą komend poniżej. Rozmiar odstępów jest mierzony w calach.
% ---------------------------------------
% MARGINS
% ---------------------------------------
\setmarginsrb
{ 0.7in} % left margin
{ 0.6in} % top margin
{ 0.7in} % right margin
{ 0.8in} % bottom margin
{  20pt} % head height
{0.25in} % head sep
{   9pt} % foot height
{ 0.3in} % foot sep



% ---------------------------------------
% Często używane pakiety
% ---------------------------------------
% \usepackage{csquotes}  % Pozwala w prosty sposób wstawiać cytaty do tekstu.
\usepackage{xcolor}  % Pozwala używać kolorowych czcionek (zapewne dużo
% więcej, ale ja nie potrafię nic o tym powiedzieć).



% ---------------------------------------
% Pakiety napisane przez użytkownika.
% Mają być w tym samym katalogu to ten plik .tex
% ---------------------------------------
\usepackage{latexgeneralcommands}



% ---------------------------------------------------------------------
% Dodatkowe ustawienia dla języka polskiego
% ---------------------------------------------------------------------
\renewcommand{\thesection}{\arabic{section}.}
% Kropki po numerach rozdziału (polski zwyczaj topograficzny)
\renewcommand{\thesubsection}{\thesection\arabic{subsection}}
% Brak kropki po numerach podrozdziału



% ---------------------------------------
% Ustawienia różnych parametrów tekstu
% ---------------------------------------
\renewcommand{\arraystretch}{1.2}  % Ustawienie szerokości odstępów między
% wierszami w tabelach.





% ---------------------------------------
% Pakiet „hyperref”
% Polecano by umieszczać go na końcu preambuły.
% ---------------------------------------
\usepackage{hyperref}  % Pozwala tworzyć hiperlinki i zamienia odwołania
% do bibliografii na hiperlinki.










% ---------------------------------------------------------------------
% Tytuł, autor, data
\title{Eseje, publicystyka~-- błędy i~uwagi}

% \author{}
% \date{}
% ---------------------------------------------------------------------










% ####################################################################
% Początek dokumentu
\begin{document}
% ####################################################################





% ######################################
\maketitle % Tytuł całego tekstu
% ######################################





% ############################
\Work{ % Autor i tytuł dzieła
  Marek Jan Chodakiewicz \\
  „O~cywilizacji śmierci. \\
  Jak zatrzymać antykulturę totalitarnych mniejszości”,
  \cite{ChodakiewiczOCywilizacjiSmierci2019} }


% ##################
\CenterBoldFont{Uwagi ogólne}


\start \textbf{Str. 11, wiersze 6, 10.} Po tych wiersza w~tekście
powinien znajdować~się odstęp.

% \vspace{\spaceFour}





% ##################
\CenterBoldFont{Błędy}


\begin{center}

  \begin{tabular}{|c|c|c|c|c|}
    \hline
    & \multicolumn{2}{c|}{} & & \\
    Strona & \multicolumn{2}{c|}{Wiersz} & Jest
                              & Powinno być \\ \cline{2-3}
    & Od góry & Od dołu & & \\
    \hline
    14  & 14 & & śmierci śmierci & śmierci \\
    16  & & 16 & Ldweicy & Lewicy \\
    43  & & 20 & \textit{Bogomil :} & \textit{Bogomil:} \\
    43  & & 14 & \textit{Inqistion} & \textit{Inquisition} \\
    53  & & 11 & Company. & Company, \\
    54  & &  4 & 1\textbf{8} & 18 \\
    54  & &  3 & \textit{Bueaty} & \textit{Beauty} \\
    62  & &  5 & Use & \textit{Use} \\
    62  & &  4 & The~History~of Sexuality
           & \textit{The~History~of Sexuality} \\
    63  &  3 & & wyjątków & z~wyjątków \\
    64  & &  2 & The~Gay Metropolis & \textit{The~Gay Metropolis} \\
    67  & 17 & & Action League & \textit{Action League} \\
    68  & & 12 & ludzkiej & ludzkiej” \\
    69  & & 14 & religi świata & Religi Świata \\
    % & & & & \\
    % & & & & \\
    % & & & & \\
    % & & & & \\
    \hline
  \end{tabular}





  % \begin{tabular}{|c|c|c|c|c|}
  %   \hline
  %   & \multicolumn{2}{c|}{} & & \\
  %   Strona & \multicolumn{2}{c|}{Wiersz} & Jest
  %   & Powinno być \\ \cline{2-3}
  %   & Od góry & Od dołu & & \\
  %   \hline
  %   %   & & & & \\
  %   %   & & & & \\
  %   %   & & & & \\
  %   %   & & & & \\
  %   %   & & & & \\
  %   %   & & & & \\
  %   %   & & & & \\
  %   %   & & & & \\
  %   %   & & & & \\
  %   %   & & & & \\
  %   %   & & & & \\
  %   %   & & & & \\
  %   %   & & & & \\
  %   %   & & & & \\
  %   %   & & & & \\
  %   %   & & & & \\
  %   %   & & & & \\
  %   %   & & & & \\
  %   %   & & & & \\
  %   %   & & & & \\
  %   %   & & & & \\
  %   %   & & & & \\
  %   %   & & & & \\
  %   %   & & & & \\
  %   %   & & & & \\
  %   %   & & & & \\
  %   %   & & & & \\
  %   %   & & & & \\
  %   %   & & & & \\
  %   %   & & & & \\
  %   %   & & & & \\
  %   %   & & & & \\
  %   %   & & & & \\
  %   %   & & & & \\
  %   %   & & & & \\
  %   %   & & & & \\
  %   %   & & & & \\
  %   %   & & & & \\
  %   \hline
  % \end{tabular}




  % \begin{tabular}{|c|c|c|c|c|}
  %   \hline
  %   & \multicolumn{2}{c|}{} & & \\
  %   Strona & \multicolumn{2}{c|}{Wiersz} & Jest
  %   & Powinno być \\ \cline{2-3}
  %   & Od góry & Od dołu & & \\
  %   \hline
  %   %   & & & & \\
  %   %   & & & & \\
  %   %   & & & & \\
  %   %   & & & & \\
  %   %   & & & & \\
  %   %   & & & & \\
  %   %   & & & & \\
  %   %   & & & & \\
  %   %   & & & & \\
  %   %   & & & & \\
  %   %   & & & & \\
  %   %   & & & & \\
  %   %   & & & & \\
  %   %   & & & & \\
  %   %   & & & & \\
  %   %   & & & & \\
  %   %   & & & & \\
  %   %   & & & & \\
  %   %   & & & & \\
  %   %   & & & & \\
  %   %   & & & & \\
  %   %   & & & & \\
  %   %   & & & & \\
  %   %   & & & & \\
  %   %   & & & & \\
  %   %   & & & & \\
  %   %   & & & & \\
  %   %   & & & & \\
  %   %   & & & & \\
  %   %   & & & & \\
  %   %   & & & & \\
  %   %   & & & & \\
  %   %   & & & & \\
  %   %   & & & & \\
  %   %   & & & & \\
  %   %   & & & & \\
  %   %   & & & & \\
  %   %   & & & & \\
  %   \hline
  % \end{tabular}

\end{center}


\noindent
\Jest  Selected Writings~of Alexandra Kollontai \\
\Powin \textit{Selected Writings~of Alexandra Kollontai} \\

% \StrWd{11}{16} \\
% \Jest
% \Powin



% \StrWd{11}{5} \\
% \Jest
% \Powin
% \StrWd{11}{1} \\
% \Jest
% \Powin
% \StrWg{12}{11} \\
% \Jest
% \Powin
% \StrWd{12}{18} \\
% \Jest
% \Powin
% \StrWd{12}{8} \\
% \Jest
% \Powin \\
% \StrWg{13}{10} \\
% \Jest
% \Powin
% \StrWd{13}{4} \\
% \Jest
% \Powin \\
% \StrWg{15}{1} \\
% \Jest
% \Powin
% \StrWg{15}{3} \\
% \Jest
% \Powin
% \StrWg{15}{10} \\
% \Jest
% \Powin
% \StrWg{43}{2} \\
% \Jest
% \Powin

\vspace{\spaceTwo}
% ############################










% ############################
\Work{ % Autor i tytuł dzieła
  Michael Hoffman \\
  „Judaizm zdemaskowany! Tom I”, \cite{} }


% ##################
\CenterBoldFont{Błędy}


\begin{center}

  \begin{tabular}{|c|c|c|c|c|}
    \hline
    & \multicolumn{2}{c|}{} & & \\
    Strona & \multicolumn{2}{c|}{Wiersz} & Jest
                              & Powinno być \\ \cline{2-3}
    & Od góry & Od dołu & & \\
    \hline
    14  &  7 & & opanowaniua & opanowania \\
    17  & 15 & & nam$^{ 7 }$ & nam \\
    20  & &  1 & $^{ 9 }$\textit{Canadian} & $^{ 8 }$\textit{Canadian} \\
    20  & &  1 & \textit{News30} & \textit{News, 30} \\
    27  & 10 & & \textit{Dona Richardsona} & Dona Richardsona \\
    27  & 11 & & \textit{Joela Richardsona} & Joela Richardsona \\
    30  & 19 & & podescytowani & podekscytowani \\
    % & & & & \\
    % & & & & \\
    % & & & & \\
    \hline
  \end{tabular}

\end{center}


\vspace{\spaceOne}
% ############################










% ############################
\Work{ % Autor i tytuł dzieła
  Zdzisław Krasnodębski \\
  „Drzemka rozsądnych. Zebrane eseje i~szkice”,
  \cite{KrasnodebskiDrzemkaRozsadnych2006} }


% ##################
\CenterBoldFont{Uwagi}


\begin{center}

  \begin{tabular}{|c|c|c|c|c|}
    \hline
    & \multicolumn{2}{c|}{} & & \\
    Strona & \multicolumn{2}{c|}{Wiersz} & Jest
                              & Powinno być \\ \cline{2-3}
    & Od góry & Od dołu & & \\
    \hline
    11  & &  5 & z OMP & OMP \\
    145 & &  3 & 1980 & 1989 \\
    168 &  1 & & których~się & której~się \\
    168 &  1 & & których żyją & której żyją \\
    223 & &  2 & \textit{Rzeczpospoliej} & \textit{Rzeczpospolitej} \\
    254 & & 15 & partią & partii \\
    310 & & 14 & zachęcić & zachęcić~go \\
    325 & 14 & & spierający & spierający~się \\
    % & & & & \\
    % & & & & \\
    % & & & & \\
    % & & & & \\
    \hline
  \end{tabular}

\end{center}


\noindent
\StrWd{72}{2} \\
\Jest  Frankreich 1871, Deutschland 1918 \\
\Powin \textit{Frankreich 1871, Deutschland 1918} \\


\vspace{\spaceTwo}
% ############################










% ############################
\Work{ % Autor i tytuł dzieła
  Zdzisław Krasnodębski \\
  „Zwycięzca po~przejściach. Zebrane eseje i~szkice~V”,
  \cite{KrasnodebskiZwyciezcaPoPrzejsciach2012} }


% ##################
\CenterBoldFont{Uwagi do~konkretnych stron}


\start \Str{317} Pominięto miejsce i~datę pierwszej publikacji
artykułu \textit{Nie udawaj Greka, Polsko!}


\vspace{\spaceTwo}
% ############################










% ############################
\Work{ % Autor i tytuł dzieła
  Andrzej Nowak \\
  „Strachy i lachy. Przemiany polskiej pamięci 1982--2012”,
  \cite{} }


% ##################
\CenterBoldFont{Uwagi do~konkretnych stron}


\start \Str{47} T.~S.~Eliot jest na tej stronie nazwany „wielkim
poetą katolickim”, acz z~tego co wiem do Kościoła nigdy nie
przyszedł, zamiast tego dołączył do jakiegoś wyznania
anglokatolickiego. Zaś użycie przymiotnika „wielki” w~odniesieniu to
tego poety, którego twórczości nie da~się czytać, jest już na~pewno
błędem.


\vspace{\spaceTwo}
% ############################










% ############################
\Work{ % Autor i tytuł dzieła
  Andrzej Nowak \\
  „Intelektualna historia III~RP. Rozmowy z~lat 1990--2012”,
  \cite{NowakIntelektualnaHistoriaIIIRP2013} }


% ##################
\CenterBoldFont{Błędy}


\begin{center}

  \begin{tabular}{|c|c|c|c|c|}
    \hline
    & \multicolumn{2}{c|}{} & & \\
    Strona & \multicolumn{2}{c|}{Wiersz} & Jest
                              & Powinno być \\ \cline{2-3}
    & Od góry & Od dołu & & \\
    \hline
    195 &  6 & & pomocą & pomocy \\
    579 & &  3 & osób. & osób, \\
    580 & & 10 & od & do \\
    665 &  7 & & The~National Interest” & „The~National Interest” \\
    % & & & & \\
    % & & & & \\
    \hline
  \end{tabular}

\end{center}


\noindent
\textbf{Przednia okładka, wiersz 14.} \\
\Jest \textit{ImperologicalStudies.APolishPerspective}(2011);
\textit{Czaswalki} \\
\Powin \textit{Imperological Studies. A Polish Perspective} (2011);
\textit{Czas walki} \\
\textbf{Przednia okładka, wiersz 10 (od dołu).} \\
\Jest  \ldots w~Brnie \\
\Powin w~Brnie \\


\vspace{\spaceTwo}
% ############################










% ############################
\Work{ % Autor i tytuł dzieła
  Andrzej Nowak \\
  „Historia i~polityka”, \cite{NowakHistoriaIPolityka2016} }


% ##################
\CenterBoldFont{Uwagi do konkretnych stron}


\start \Str{} Nowak popełni tu pewien błąd pisząc o~grze komputerowej
„Dzikie Pola”, jest to standardowa stołowa gra RPG i~nie ma nic
wspólnego z~komputerem.





% ##################
\CenterBoldFont{Błędy}


\begin{center}

  \begin{tabular}{|c|c|c|c|c|}
    \hline
    & \multicolumn{2}{c|}{} & & \\
    Strona & \multicolumn{2}{c|}{Wiersz} & Jest
                              & Powinno być \\ \cline{2-3}
    & Od góry & Od dołu & & \\
    \hline
    6   &  2 & & 448 & 484 \\
    18  & 18 & & „ kult & „kult \\
    % & & & & \\
    % & & & & \\
    % & & & & \\
    % & & & & \\
    99  & &  2 & (1918--1920) & (1918--2008) \\
    133 & & 23 & 1981 & 1918 \\
    135 & &  5 & 361 (przytoczony & 361. Przytoczony \\
    % & & & & \\
    % & & & & \\
    \hline
  \end{tabular}

\end{center}


\vspace{\spaceTwo}
% ############################








% ######################################
\section{Eseje filozoficzne}
% ######################################



% ############################
\Work{ % Autor i tytuł dzieła
  Roger Scruton \\
  „Zielona filozofia \\
  Jak poważnie myśleć o~naszej planecie”,
  \cite{ChodakiewiczOCywilizacjiSmierci2019} }


% ##################
\CenterBoldFont{Uwagi}


\start \StrWg{50}{6} Możliwe, że fragment ,,pozostaną w~lęku
i~z~drżeniem'', lepiej byłoby przetłumaczyć jak ,,poczują lęk
i~trwogę''.

\vspace{\spaceFour}





% ##################
\CenterBoldFont{Błędy}


\begin{center}

  \begin{tabular}{|c|c|c|c|c|}
    \hline
    & \multicolumn{2}{c|}{} & & \\
    Strona & \multicolumn{2}{c|}{Wiersz} & Jest
                              & Powinno być \\ \cline{2-3}
    & Od góry & Od dołu & & \\
    \hline
    14  & &  9 & \textit{i} & \textit{and} \\
    18  & &  4 & 2008 & 2008) \\
    % & & & & \\
    % & & & & \\
    % & & & & \\
    % & & & & \\
    % & & & & \\
    % & & & & \\
    \hline
  \end{tabular}





  % \begin{tabular}{|c|c|c|c|c|}
  %   \hline
  %   & \multicolumn{2}{c|}{} & & \\
  %   Strona & \multicolumn{2}{c|}{Wiersz} & Jest
  %   & Powinno być \\ \cline{2-3}
  %   & Od góry & Od dołu & & \\
  %   \hline
  %   %   & & & & \\
  %   %   & & & & \\
  %   %   & & & & \\
  %   %   & & & & \\
  %   %   & & & & \\
  %   %   & & & & \\
  %   %   & & & & \\
  %   %   & & & & \\
  %   %   & & & & \\
  %   %   & & & & \\
  %   %   & & & & \\
  %   %   & & & & \\
  %   %   & & & & \\
  %   %   & & & & \\
  %   %   & & & & \\
  %   %   & & & & \\
  %   %   & & & & \\
  %   %   & & & & \\
  %   %   & & & & \\
  %   %   & & & & \\
  %   %   & & & & \\
  %   %   & & & & \\
  %   %   & & & & \\
  %   %   & & & & \\
  %   %   & & & & \\
  %   %   & & & & \\
  %   %   & & & & \\
  %   %   & & & & \\
  %   %   & & & & \\
  %   %   & & & & \\
  %   %   & & & & \\
  %   %   & & & & \\
  %   %   & & & & \\
  %   %   & & & & \\
  %   %   & & & & \\
  %   %   & & & & \\
  %   %   & & & & \\
  %   %   & & & & \\
  %   \hline
  % \end{tabular}





  % \begin{tabular}{|c|c|c|c|c|}
  %   \hline
  %   & \multicolumn{2}{c|}{} & & \\
  %   Strona & \multicolumn{2}{c|}{Wiersz} & Jest
  %   & Powinno być \\ \cline{2-3}
  %   & Od góry & Od dołu & & \\
  %   \hline
  %   %   & & & & \\
  %   %   & & & & \\
  %   %   & & & & \\
  %   %   & & & & \\
  %   %   & & & & \\
  %   %   & & & & \\
  %   %   & & & & \\
  %   %   & & & & \\
  %   %   & & & & \\
  %   %   & & & & \\
  %   %   & & & & \\
  %   %   & & & & \\
  %   %   & & & & \\
  %   %   & & & & \\
  %   %   & & & & \\
  %   %   & & & & \\
  %   %   & & & & \\
  %   %   & & & & \\
  %   %   & & & & \\
  %   %   & & & & \\
  %   %   & & & & \\
  %   %   & & & & \\
  %   %   & & & & \\
  %   %   & & & & \\
  %   %   & & & & \\
  %   %   & & & & \\
  %   %   & & & & \\
  %   %   & & & & \\
  %   %   & & & & \\
  %   %   & & & & \\
  %   %   & & & & \\
  %   %   & & & & \\
  %   %   & & & & \\
  %   %   & & & & \\
  %   %   & & & & \\
  %   %   & & & & \\
  %   %   & & & & \\
  %   %   & & & & \\
  %   \hline
  % \end{tabular}

\end{center}


% \noindent
% \StrWd{11}{16} \\
% \Jest
% \Powin



% \StrWd{11}{5} \\
% \Jest
% \Powin
% \StrWd{11}{1} \\
% \Jest
% \Powin
% \StrWg{12}{11} \\
% \Jest
% \Powin
% \StrWd{12}{18} \\
% \Jest
% \Powin
% \StrWd{12}{8} \\
% \Jest
% \Powin \\
% \StrWg{13}{10} \\
% \Jest
% \Powin
% \StrWd{13}{4} \\
% \Jest
% \Powin
% \StrWg{15}{1} \\
% \Jest
% \Powin
% \StrWg{15}{3} \\
% \Jest
% \Powin
% \StrWg{15}{10} \\
% \Jest
% \Powin
% \StrWg{43}{2} \\
% \Jest
% \Powin


\vspace{\spaceTwo}
% ############################










% ####################################################################
% ####################################################################
% Bibliografia
\bibliographystyle{plalpha}

\bibliography{VariousFieldsBooks}{}





% ############################

% Koniec dokumentu
\end{document}