% ---------------------------------------------------------------------
% Basic configuration and packages
% ---------------------------------------------------------------------
% Package for discovering wrong and outdated usage of LaTeX.
% More information to be found in l2tabu English version.
\RequirePackage[l2tabu, orthodox]{nag}
% Class of LaTeX document: {size of paper, size of font}[document class]
\documentclass[a4paper,11pt]{article}



% ---------------------------------------
% Packages not tied to particular normal language
% ---------------------------------------
% This package should improved spaces in the text.
\usepackage{microtype}
% Add few important symbols, like text Celcius degree
\usepackage{textcomp}



% ---------------------------------------
% Polonization of LaTeX document
% ---------------------------------------
% Basic polonization of the text
\usepackage[MeX]{polski}
% Switching on UTF-8 encoding
\usepackage[utf8]{inputenc}
% Adding font Latin Modern
\usepackage{lmodern}
% Package is need for fonts Latin Modern
\usepackage[T1]{fontenc}



% ---------------------------------------
% Setting margins
% ---------------------------------------
\usepackage[a4paper, total={14cm, 25cm}]{geometry}



% ---------------------------------------
% Setting vertical spaces in the text
% ---------------------------------------
% Setting space between lines
\renewcommand{\baselinestretch}{1.1}

% Setting space between lines in tables
\renewcommand{\arraystretch}{1.4}





% ------------------------------
% Private packages
% You need to put them in the same directory as .tex file
% ------------------------------
% Contains various command useful for working with a text
\usepackage{latexgeneralcommands}





% ------------------------------
% Package ``hyperref''
% They advised to put it on the end of preambule
% ------------------------------
% It allows you to use hyperlinks in the text
\usepackage{hyperref}










% ---------------------------------------------------------------------
% Tytuł i autor tekstu
\title{Komiksy, powieści graficzne \\
  {\Large Błędy i~uwagi}}

\author{Kamil Ziemian}


% \date{}
% ---------------------------------------------------------------------










% ####################################################################
% Początek dokumentu
\begin{document}
% ####################################################################





% ######################################
\maketitle % Tytuł całego tekstu
% ######################################





% ######################################
\section{Komiksy}
% ######################################



% ##############################
\Work{ % Autor i tytuł dzieła
  Will Eisner \\
  \textit{Spirit. Najlepsze opowieści}, \cite{EisnerSpirit2009}}


% ##################
\CenterBoldFont{Błędy}


\noindent
\textbf{Str. 13, obrazek w~lewym górnym roku.}

% \begin{center}
%   \begin{tabular}{|c|c|c|c|c|}
%     \hline
%     & \multicolumn{2}{c|}{} & & \\
      %       Strona & \multicolumn{2}{c|}{Wiersz} & Jest
% & Powinno być \\ \cline{2-3}
      %       & Od góry & Od dołu & & \\
% \hline
%     & & & & \\
%     & & & & \\
%     & & & & \\
%     & & & & \\ \hline
%   \end{tabular}
% \end{center}
% \noi \\
% \tb{Str. 13, obrazek w~lewym górnym roku.} \\
% \Jest stronie 68 \\
% \Pow stronie 62 \\


\vspace{\spaceTwo}
% ############################










% ######################################
\section{Powieści graficzne}
% ######################################



% ############################
\Work{ % Autor i tytuł dzieła
  Will Eisner \\
  \textit{Życie w~obrazkach. Opowieści autobiograficzne},
  \cite{EisnerZycieWObrazkach2009}}


% ##################
\CenterBoldFont{Błędy}


\noindent \\
\textbf{Str. 92, I kolumna, wiersz 27.} \\
\Jest  stronie 68 \\
\Powin stronie 62 \\


\vspace{\spaceTwo}
% ############################










% ####################################################################
% ####################################################################
% Bibliografia

\bibliographystyle{plalpha}

\bibliography{VisualArtsBooks}{}





% ############################

% Koniec dokumentu
\end{document}
