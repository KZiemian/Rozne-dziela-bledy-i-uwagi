% ------------------------------------------------------------------------------------------------------------------
% Basic configuration and packages
% ------------------------------------------------------------------------------------------------------------------
% Package for discovering wrong and outdated usage of LaTeX.
% More information to be found in l2tabu English version.
\RequirePackage[l2tabu, orthodox]{nag}
% Class of LaTeX document: {size of paper, size of font}[document class]
\documentclass[a4paper,11pt]{article}



% ------------------------------------------------------
% Packages not tied to particular normal language
% ------------------------------------------------------
% This package should improved spaces in the text
\usepackage{microtype}
% Add few important symbols, like text Celcius degree
\usepackage{textcomp}



% ------------------------------------------------------
% Polonization of LaTeX document
% ------------------------------------------------------
% Basic polonization of the text
\usepackage[MeX]{polski}
% Switching on UTF-8 encoding
\usepackage[utf8]{inputenc}
% Adding font Latin Modern
\usepackage{lmodern}
% Package is need for fonts Latin Modern
\usepackage[T1]{fontenc}



% ------------------------------------------------------
% Setting margins
% ------------------------------------------------------
\usepackage[a4paper, total={14cm, 25cm}]{geometry}



% ------------------------------------------------------
% Setting vertical spaces in the text
% ------------------------------------------------------
% Setting space between lines
\renewcommand{\baselinestretch}{1.1}

% Setting space between lines in tables
\renewcommand{\arraystretch}{1.4}



% ------------------------------------------------------
% Packages for scientific papers
% ------------------------------------------------------
% Switching off \lll symbol, that I guess is representing letter "Ł"
% It collide with `amsmath' package's command with the same name
\let\lll\undefined
% Basic package from American Mathematical Society (AMS)
\usepackage[intlimits]{amsmath}
% Equations are numbered separately in every section
\numberwithin{equation}{section}

% Other very useful packages from AMS
\usepackage{amsfonts}
\usepackage{amssymb}
\usepackage{amscd}
\usepackage{amsthm}

% Package with better looking calligraphy fonts
\usepackage{calrsfs}

% Package with better looking greek letters
% Example of use: pi -> \uppi
\usepackage{upgreek}
% Improving look of lambda letter
\let\oldlambda\Lambda
\renewcommand{\lambda}{\uplambda}




% ------------------------------------------------------
% BibLaTeX
% ------------------------------------------------------
% Package biblatex, with biber as its backend, allow us to handle
% bibliography entries that use Unicode symbols outside ASCII
\usepackage[
language=polish,
backend=biber,
style=alphabetic,
url=false,
eprint=true,
]{biblatex}

\addbibresource{Eseje-publicystyka-etc-Bibliography.bib}





% ------------------------------------------------------
% Defining new environments (?)
% ------------------------------------------------------
% Defining enviroment "Wniosek"
\newtheorem{corollary}{Wniosek}
\newtheorem{definition}{Definicja}
\newtheorem{theorem}{Twierdzenie}





% ------------------------------------------------------
% Local packages
% You need to put them in the same directory as .tex file
% ------------------------------------------------------
% Package containing various command useful for working with a text
\usepackage{general-commands}





% ------------------------------------------------------
% Package "hyperref"
% They advised to put it on the end of preambule
% ------------------------------------------------------
% It allows you to use hyperlinks in the text
\usepackage{hyperref}










% ------------------------------------------------------------------------------------------------------------------
% Title and author of the text
\title{Eseje, publicystyka, etc. \\
  {\Large Błędy i~uwagi}}

\author{Kamil Ziemian}


% \date{}
% ------------------------------------------------------------------------------------------------------------------










% ####################################################################
% Początek dokumentu
\begin{document}
% ####################################################################





% ######################################
\maketitle % Tytuł całego tekstu
% ######################################





% ############################
\section{Marek Jan Chodakiewicz
  \textit{O~cywilizacji śmierci. Jak zatrzymać antykulturę
    totalitarnych mniejszości},
  \cite{ChodakiewiczOCywilizacjiSmierci2019}}


% ##################
\CenterBoldFont{Uwagi ogólne}


\noindent
\textbf{Str. 11, wiersze 6, 10.} Po tych wiersza w~tekście
powinien znajdować~się odstęp.

% \vspace{\spaceFour}





% ##################
\CenterBoldFont{Błędy}


\begin{center}

  \begin{tabular}{|c|c|c|c|c|}
    \hline
    & \multicolumn{2}{c|}{} & & \\
    Strona & \multicolumn{2}{c|}{Wiersz} & Jest
                              & Powinno być \\ \cline{2-3}
    & Od góry & Od dołu & & \\
    \hline
    14  & 14 & & śmierci śmierci & śmierci \\
    16  & & 16 & Ldweicy & Lewicy \\
    43  & & 20 & \textit{Bogomil :} & \textit{Bogomil:} \\
    43  & & 14 & \textit{Inqistion} & \textit{Inquisition} \\
    53  & & 11 & Company. & Company, \\
    54  & &  4 & 1\textbf{8} & 18 \\
    54  & &  3 & \textit{Bueaty} & \textit{Beauty} \\
    62  & &  5 & Use & \textit{Use} \\
    62  & &  4 & The~History~of Sexuality
           & \textit{The~History~of Sexuality} \\
    63  &  3 & & wyjątków & z~wyjątków \\
    64  & &  2 & The~Gay Metropolis & \textit{The~Gay Metropolis} \\
    67  & 17 & & Action League & \textit{Action League} \\
    68  & & 12 & ludzkiej & ludzkiej” \\
    69  & & 14 & religi świata & Religi Świata \\
    % & & & & \\
    % & & & & \\
    % & & & & \\
    % & & & & \\
    \hline
  \end{tabular}





  % \begin{tabular}{|c|c|c|c|c|}
  %   \hline
  %   & \multicolumn{2}{c|}{} & & \\
  %   Strona & \multicolumn{2}{c|}{Wiersz} & Jest
  %   & Powinno być \\ \cline{2-3}
  %   & Od góry & Od dołu & & \\
  %   \hline
  %   %   & & & & \\
  %   %   & & & & \\
  %   %   & & & & \\
  %   %   & & & & \\
  %   %   & & & & \\
  %   %   & & & & \\
  %   %   & & & & \\
  %   %   & & & & \\
  %   %   & & & & \\
  %   %   & & & & \\
  %   %   & & & & \\
  %   %   & & & & \\
  %   %   & & & & \\
  %   %   & & & & \\
  %   %   & & & & \\
  %   %   & & & & \\
  %   %   & & & & \\
  %   %   & & & & \\
  %   %   & & & & \\
  %   %   & & & & \\
  %   %   & & & & \\
  %   %   & & & & \\
  %   %   & & & & \\
  %   %   & & & & \\
  %   %   & & & & \\
  %   %   & & & & \\
  %   %   & & & & \\
  %   %   & & & & \\
  %   %   & & & & \\
  %   %   & & & & \\
  %   %   & & & & \\
  %   %   & & & & \\
  %   %   & & & & \\
  %   %   & & & & \\
  %   %   & & & & \\
  %   %   & & & & \\
  %   %   & & & & \\
  %   %   & & & & \\
  %   \hline
  % \end{tabular}




  % \begin{tabular}{|c|c|c|c|c|}
  %   \hline
  %   & \multicolumn{2}{c|}{} & & \\
  %   Strona & \multicolumn{2}{c|}{Wiersz} & Jest
  %   & Powinno być \\ \cline{2-3}
  %   & Od góry & Od dołu & & \\
  %   \hline
  %   %   & & & & \\
  %   %   & & & & \\
  %   %   & & & & \\
  %   %   & & & & \\
  %   %   & & & & \\
  %   %   & & & & \\
  %   %   & & & & \\
  %   %   & & & & \\
  %   %   & & & & \\
  %   %   & & & & \\
  %   %   & & & & \\
  %   %   & & & & \\
  %   %   & & & & \\
  %   %   & & & & \\
  %   %   & & & & \\
  %   %   & & & & \\
  %   %   & & & & \\
  %   %   & & & & \\
  %   %   & & & & \\
  %   %   & & & & \\
  %   %   & & & & \\
  %   %   & & & & \\
  %   %   & & & & \\
  %   %   & & & & \\
  %   %   & & & & \\
  %   %   & & & & \\
  %   %   & & & & \\
  %   %   & & & & \\
  %   %   & & & & \\
  %   %   & & & & \\
  %   %   & & & & \\
  %   %   & & & & \\
  %   %   & & & & \\
  %   %   & & & & \\
  %   %   & & & & \\
  %   %   & & & & \\
  %   %   & & & & \\
  %   %   & & & & \\
  %   \hline
  % \end{tabular}

\end{center}

\VerSpaceTwo


\noindent
\Jest Selected Writings~of Alexandra Kollontai \\
\PowinnoByc \textit{Selected Writings~of Alexandra Kollontai} \\

% \StrWd{11}{16} \\
% \Jest
% \Powin



% \StrWd{11}{5} \\
% \Jest
% \Powin
% \StrWd{11}{1} \\
% \Jest
% \Powin
% \StrWg{12}{11} \\
% \Jest
% \Powin
% \StrWd{12}{18} \\
% \Jest
% \Powin
% \StrWd{12}{8} \\
% \Jest
% \Powin \\
% \StrWg{13}{10} \\
% \Jest
% \Powin
% \StrWd{13}{4} \\
% \Jest
% \Powin \\
% \StrWg{15}{1} \\
% \Jest
% \Powin
% \StrWg{15}{3} \\
% \Jest
% \Powin
% \StrWg{15}{10} \\
% \Jest
% \Powin
% \StrWg{43}{2} \\
% \Jest
% \Powin


% ############################










% ############################
\section{ % Autor i tytuł dzieła
  Michael Hoffman \\
  \textit{Judaizm zdemaskowany! Tom~I}, \cite{}}


% ##################
\CenterBoldFont{Błędy}


\begin{center}

  \begin{tabular}{|c|c|c|c|c|}
    \hline
    & \multicolumn{2}{c|}{} & & \\
    Strona & \multicolumn{2}{c|}{Wiersz} & Jest
                              & Powinno być \\ \cline{2-3}
    & Od góry & Od dołu & & \\
    \hline
    14  &  7 & & opanowaniua & opanowania \\
    17  & 15 & & nam$^{ 7 }$ & nam \\
    20  & &  1 & $^{ 9 }$\textit{Canadian} & $^{ 8 }$\textit{Canadian} \\
    20  & &  1 & \textit{News30} & \textit{News, 30} \\
    27  & 10 & & \textit{Dona Richardsona} & Dona Richardsona \\
    27  & 11 & & \textit{Joela Richardsona} & Joela Richardsona \\
    30  & 19 & & podescytowani & podekscytowani \\
    % & & & & \\
    % & & & & \\
    % & & & & \\
    \hline
  \end{tabular}

\end{center}

\VerSpaceTwo

% ############################










% ############################
\section{ % Autor i tytuł dzieła
  Zdzisław Krasnodębski \\
  \textit{Drzemka rozsądnych. Zebrane eseje i~szkice},
  \cite{KrasnodebskiDrzemkaRozsadnych2006}}


% ##################
\CenterBoldFont{Uwagi}


\begin{center}

  \begin{tabular}{|c|c|c|c|c|}
    \hline
    & \multicolumn{2}{c|}{} & & \\
    Strona & \multicolumn{2}{c|}{Wiersz} & Jest
                              & Powinno być \\ \cline{2-3}
    & Od góry & Od dołu & & \\
    \hline
    11  & &  5 & z OMP & OMP \\
    145 & &  3 & 1980 & 1989 \\
    168 &  1 & & których~się & której~się \\
    168 &  1 & & których żyją & której żyją \\
    223 & &  2 & \textit{Rzeczpospoliej} & \textit{Rzeczpospolitej} \\
    254 & & 15 & partią & partii \\
    310 & & 14 & zachęcić & zachęcić~go \\
    325 & 14 & & spierający & spierający~się \\
    % & & & & \\
    % & & & & \\
    % & & & & \\
    % & & & & \\
    \hline
  \end{tabular}

\end{center}

\VerSpaceTwo


\noindent
\StrWierszDol{72}{2} \\
\Jest Frankreich 1871, Deutschland 1918 \\
\PowinnoByc \textit{Frankreich 1871, Deutschland 1918} \\



% ############################










% ############################
\section{ % Autor i tytuł dzieła
  Zdzisław Krasnodębski \\
  \textit{Zwycięzca po~przejściach. Zebrane eseje i~szkice~V},
  \cite{KrasnodebskiZwyciezcaPoPrzejsciach2012} }


% ##################
\CenterBoldFont{Uwagi do~konkretnych stron}


\Str{317} Pominięto miejsce i~datę pierwszej publikacji
artykułu \textit{Nie udawaj Greka, Polsko!}

\VerSpaceFour

% ############################










% ############################
\section{ % Autor i tytuł dzieła
  Andrzej Nowak \\
  \textit{Strachy i lachy. Przemiany polskiej pamięci 1982--2012},
  \cite{}}


% ##################
\CenterBoldFont{Uwagi do~konkretnych stron}


\Str{47} T.~S.~Eliot jest na tej stronie nazwany „wielkim
poetą katolickim”, acz z~tego co wiem do Kościoła nigdy nie
przyszedł, zamiast tego dołączył do jakiegoś wyznania
anglokatolickiego. Zaś użycie przymiotnika „wielki” w~odniesieniu to
tego poety, którego twórczości nie da~się czytać, jest już na~pewno
błędem.



% ############################










% ############################
\section{ % Autor i tytuł dzieła
  Andrzej Nowak \\
  \textit{Intelektualna historia III~RP. Rozmowy z~lat 1990--2012},
  \cite{NowakIntelektualnaHistoriaIIIRP2013}}


% ##################
\CenterBoldFont{Błędy}


\begin{center}

  \begin{tabular}{|c|c|c|c|c|}
    \hline
    & \multicolumn{2}{c|}{} & & \\
    Strona & \multicolumn{2}{c|}{Wiersz} & Jest
                              & Powinno być \\ \cline{2-3}
    & Od góry & Od dołu & & \\
    \hline
    195 &  6 & & pomocą & pomocy \\
    579 & &  3 & osób. & osób, \\
    580 & & 10 & od & do \\
    665 &  7 & & The~National Interest” & „The~National Interest” \\
    % & & & & \\
    % & & & & \\
    \hline
  \end{tabular}

\end{center}

\VerSpaceTwo


\noindent
\textbf{Przednia okładka, wiersz 14.} \\
\Jest \textit{ImperologicalStudies.APolishPerspective}(2011);
\textit{Czaswalki} \\
\PowinnoByc \textit{Imperological Studies. A Polish Perspective} (2011);
\textit{Czas walki} \\
\textbf{Przednia okładka, wiersz 10 (od dołu).} \\
\Jest \ldots w~Brnie \\
\PowinnoByc w~Brnie \\



% ############################










% ######################################
\section{Andrzej Nowak \textit{Historia i~polityka},
  \parencite{Nowak-Od-Polski-do-Postpolityki-Pub-2011}}
% ######################################


% ##################
\CenterBoldFont{Uwagi do konkretnych stron}


\Str{} Nowak popełni tu pewien błąd pisząc o~grze komputerowej
„Dzikie Pola”, jest to standardowa stołowa gra RPG i~nie ma nic
wspólnego z~komputerem.





% ##################
\CenterBoldFont{Błędy}


\begin{center}

  \begin{tabular}{|c|c|c|c|c|}
    \hline
    & \multicolumn{2}{c|}{} & & \\
    Strona & \multicolumn{2}{c|}{Wiersz} & Jest
                              & Powinno być \\ \cline{2-3}
    & Od góry & Od dołu & & \\
    \hline
    ???6   &  2 & & 448 & 484 \\
    ???18  & 18 & & „ kult & „kult \\
    % & & & & \\
    % & & & & \\
    % & & & & \\
    % & & & & \\
???    99  & &  2 & (1918--1920) & (1918--2008) \\
   ??? 133 & & 23 & 1981 & 1918 \\
???    135 & &  5 & 361 (przytoczony & 361. Przytoczony \\
    % & & & & \\
    % & & & & \\
    \hline
  \end{tabular}

\end{center}

\VerSpaceTwo

% ############################








% ######################################
\section{Eseje filozoficzne}
% ######################################



% ############################
\section{ % Autor i tytuł dzieła
  Roger Scruton \\
  \textit{Zielona filozofia} \\
  \textit{Jak poważnie myśleć o~naszej planecie},
  \cite{ChodakiewiczOCywilizacjiSmierci2019}}


% ##################
\CenterBoldFont{Uwagi}


\noindent
\StrWierszGora{50}{6} Możliwe, że fragment „pozostaną w~lęku
i~z~drżeniem”, lepiej byłoby przetłumaczyć jak „poczują lęk
i~trwogę”.

\VerSpaceFour





% ##################
\CenterBoldFont{Błędy}


\begin{center}

  \begin{tabular}{|c|c|c|c|c|}
    \hline
    & \multicolumn{2}{c|}{} & & \\
    Strona & \multicolumn{2}{c|}{Wiersz} & Jest
                              & Powinno być \\ \cline{2-3}
    & Od góry & Od dołu & & \\
    \hline
    14  & &  9 & \textit{i} & \textit{and} \\
    18  & &  4 & 2008 & 2008) \\
    % & & & & \\
    % & & & & \\
    % & & & & \\
    % & & & & \\
    % & & & & \\
    % & & & & \\
    \hline
  \end{tabular}





  % \begin{tabular}{|c|c|c|c|c|}
  %   \hline
  %   & \multicolumn{2}{c|}{} & & \\
  %   Strona & \multicolumn{2}{c|}{Wiersz} & Jest
  %   & Powinno być \\ \cline{2-3}
  %   & Od góry & Od dołu & & \\
  %   \hline
  %   %   & & & & \\
  %   %   & & & & \\
  %   %   & & & & \\
  %   %   & & & & \\
  %   %   & & & & \\
  %   %   & & & & \\
  %   %   & & & & \\
  %   %   & & & & \\
  %   %   & & & & \\
  %   %   & & & & \\
  %   %   & & & & \\
  %   %   & & & & \\
  %   %   & & & & \\
  %   %   & & & & \\
  %   %   & & & & \\
  %   %   & & & & \\
  %   %   & & & & \\
  %   %   & & & & \\
  %   %   & & & & \\
  %   %   & & & & \\
  %   %   & & & & \\
  %   %   & & & & \\
  %   %   & & & & \\
  %   %   & & & & \\
  %   %   & & & & \\
  %   %   & & & & \\
  %   %   & & & & \\
  %   %   & & & & \\
  %   %   & & & & \\
  %   %   & & & & \\
  %   %   & & & & \\
  %   %   & & & & \\
  %   %   & & & & \\
  %   %   & & & & \\
  %   %   & & & & \\
  %   %   & & & & \\
  %   %   & & & & \\
  %   %   & & & & \\
  %   \hline
  % \end{tabular}





  % \begin{tabular}{|c|c|c|c|c|}
  %   \hline
  %   & \multicolumn{2}{c|}{} & & \\
  %   Strona & \multicolumn{2}{c|}{Wiersz} & Jest
  %   & Powinno być \\ \cline{2-3}
  %   & Od góry & Od dołu & & \\
  %   \hline
  %   %   & & & & \\
  %   %   & & & & \\
  %   %   & & & & \\
  %   %   & & & & \\
  %   %   & & & & \\
  %   %   & & & & \\
  %   %   & & & & \\
  %   %   & & & & \\
  %   %   & & & & \\
  %   %   & & & & \\
  %   %   & & & & \\
  %   %   & & & & \\
  %   %   & & & & \\
  %   %   & & & & \\
  %   %   & & & & \\
  %   %   & & & & \\
  %   %   & & & & \\
  %   %   & & & & \\
  %   %   & & & & \\
  %   %   & & & & \\
  %   %   & & & & \\
  %   %   & & & & \\
  %   %   & & & & \\
  %   %   & & & & \\
  %   %   & & & & \\
  %   %   & & & & \\
  %   %   & & & & \\
  %   %   & & & & \\
  %   %   & & & & \\
  %   %   & & & & \\
  %   %   & & & & \\
  %   %   & & & & \\
  %   %   & & & & \\
  %   %   & & & & \\
  %   %   & & & & \\
  %   %   & & & & \\
  %   %   & & & & \\
  %   %   & & & & \\
  %   \hline
  % \end{tabular}

\end{center}

\VerSpaceTwo


% \noindent
% \StrWd{11}{16} \\
% \Jest
% \Powin



% \StrWd{11}{5} \\
% \Jest
% \Powin
% \StrWd{11}{1} \\
% \Jest
% \Powin
% \StrWg{12}{11} \\
% \Jest
% \Powin
% \StrWd{12}{18} \\
% \Jest
% \Powin
% \StrWd{12}{8} \\
% \Jest
% \Powin \\
% \StrWg{13}{10} \\
% \Jest
% \Powin
% \StrWd{13}{4} \\
% \Jest
% \Powin
% \StrWg{15}{1} \\
% \Jest
% \Powin
% \StrWg{15}{3} \\
% \Jest
% \Powin
% \StrWg{15}{10} \\
% \Jest
% \Powin
% \StrWg{43}{2} \\
% \Jest
% \Powin

% ############################










% ####################################################################
% ####################################################################
% Bibliography

\printbibliography





% ############################
% End of the document

\end{document}
